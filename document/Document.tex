\documentclass[12pt,a4paper,onecolumn]{article}

%%%%%%%%%%%%%%%%%%%%%%%%%%%%%%%%%%%
%          				PACKAGES  				              %
%%%%%%%%%%%%%%%%%%%%%%%%%%%%%%%%%%%

\usepackage[margin=1in]{geometry}
\usepackage{authblk}
\usepackage[latin1]{inputenc}
\usepackage{amsfonts}
\usepackage{a4wide,graphicx,color}
\usepackage{amsmath}
\usepackage{amssymb}
\usepackage[table]{xcolor}
\usepackage{setspace}
\usepackage{booktabs}
\usepackage{dcolumn}
\usepackage{rotating}
\usepackage{color,soul}
\usepackage{threeparttable}
\usepackage[capposition=top]{floatrow}
\usepackage[labelsep=period]{caption}

\usepackage{subcaption}
\usepackage{lscape}
\usepackage{pdflscape}
\usepackage{multicol}
\usepackage[bottom]{footmisc}
\setlength\footnotemargin{5pt}
\usepackage{longtable} %for long tables

\usepackage{enumerate}
\usepackage{units}  %nicefraction
\usepackage{placeins}
\usepackage{booktabs,multirow}
%% BibTeX settings
\usepackage{natbib}
\bibliographystyle{apalike}
%\bibliographystyle{unsrtnat}
\bibpunct{(}{)}{,}{a}{,}{,}


%% paragraph formatting
\renewcommand{\baselinestretch}{1}


% Defines columns for tables
\usepackage{array}
\newcolumntype{L}[1]{>{\raggedright\let\newline\\\arraybackslash\hspace{0pt}}m{#1}}
\newcolumntype{C}[1]{>{\centering\let\newline\\\arraybackslash\hspace{0pt}}m{#1}}
\newcolumntype{R}[1]{>{\raggedleft\let\newline\\\arraybackslash\hspace{0pt}}m{#1}}

\usepackage{comment} %to comment entire sections

\usepackage{xfrac} %sideways fractions


\usepackage{bbold} %for indicators

\setcounter{secnumdepth}{6}  %To get paragraphs referenced 

\usepackage{titlesec} %subsection smaller
\titleformat*{\subsection}{\normalsize \bfseries} %subsection smaller
%\usepackage[raggedright]{titlesec} % for sections does not hyphen words


\usepackage[colorlinks=true,linkcolor=black,urlcolor=blue,citecolor=blue]{hyperref}  %Load last
%% markup commands for code/software
\let\code=\texttt
\let\pkg=\textbf
\let\proglang=\textsf
\newcommand{\file}[1]{`\code{#1}'}
\newcommand{\email}[1]{\href{mailto:#1}{\normalfont\texttt{#1}}}
\urlstyle{same}

%%%%%%%%%%%%%%%%%%%%%%%%%%%%%%%%%%%
%     			TITLE, AUTHORS AND DATE    			  %
%%%%%%%%%%%%%%%%%%%%%%%%%%%%%%%%%%%
%% Title, authors and date

\title{Tesis PEG Template }

\author{Ignacio Sarmiento-Barbieri } 
\date{\today}
\begin{document}



\maketitle

\thispagestyle{empty} % Leaves first page without page number

%%%%%%%%%%%%%%%%%%%%%%%%%%%%%%%%%%%
%          			  ABSTRACT       					      %
%%%%%%%%%%%%%%%%%%%%%%%%%%%%%%%%%%%




\begin{abstract}
%100-word Abstract
\noindent %

\end{abstract}




% Keywords and JEL Classification
\medskip

\begin{flushleft}
	{\bf Key words: } 				\\
	{\bf JEL Classification: }
\end{flushleft}

% Ends title page and defines spacing for the rest of the document
\pagebreak
\doublespacing

%%%%%%%%%%%%%%%%%%%%%%%%%%%%%%%%%%%
%    DOCUMENT    		          %
%%%%%%%%%%%%%%%%%%%%%%%%%%%%%%%%%%%




\section{Introduction} \label{sec:intro}

Here you have to position your paper, remember RAP:
\begin{enumerate}
    \item R: research question
    \item A: your answer
    \item P: positioning in the existing literature
\end{enumerate}

\section{Literature Review}

A couple of things here. First, I prefer not to have a separate section, I think it's better to have it as part of the introduction, where you can cite the papers to position yours, and also in contrasting your contribution. You can also fold this when discussion your results, so you can compare/contrast with the existing literature


If you decide to have a separate section, please be careful in relating it to your research. I encurage you to read and follow \cite{nikolov2020writing}\footnote{This is available in the \href{https://ignaciomsarmiento.github.io/teaching/Tesis.html}{course website}} excellent advice:

\begin{quote}
    {\it ``Do not title your literature review section ``literature review''! It is a bit sophomoric. Instead, integrate your discussion of previous literature under the common thread of previous work as it relates to your main thesis.  For example if your paper is ``Do Traditional Institutions Constrain Female Entrepreneurship?'' you might want to call your literature review ``Gender norms in India''. In other words, tell your readers what is in the section...''} \citep{nikolov2020writing}
\end{quote}



\section{Data}

\section{Identification}

Writing equations is also quite easy: 

\begin{equation}
\label{eq:ols}
    \hat{\beta}= (X'X)^{-1}X'y
\end{equation}

Equation \eqref{eq:ols} shows ...

\section{Results}

Referencing tables is very easy you can do the following: In Table \ref{tab:avdiscriminationrates} ...

Citing papers is easier, for example \cite{albouy2020unlocking} shows that crime around parks. For many papers in parenthesis \cite{albouy2020unlocking,mcmillen2019more} 

\subsection{Robustness}
\section{Conclusion}





%%%%%%%%%%%%%%%%%%%%%%%%%%%%%%%%%%%
%		  References				  %
%%%%%%%%%%%%%%%%%%%%%%%%%%%%%%%%%%%

\pagebreak
\singlespacing
\bibliography{References.bib}
\pagebreak


%%%%%%%%%%%%%%%%%%%%%%%%%%%%%%%%%%%
%		  TABLES				  %
%%%%%%%%%%%%%%%%%%%%%%%%%%%%%%%%%%%
\section*{Tables and Figures}


\begin{table}[H]                                 
\footnotesize \centering                                 \begin{threeparttable}                                 \captionsetup{justification=centering}                      \caption{Overall Response Rates }                               \label{tab:avdiscriminationrates}                               \begin{tabular}{@{\extracolsep{5pt}} lcc}                       \\[-1.8ex]\hline                                  
\hline \\[-1.8ex]                                  & \multicolumn{2}{c}{\it Dependent variable:} \\                                 & \multicolumn{2}{c}{\it  Response} \\                                 \cline{2-3}\\ [-1.8ex]        
                    &\multicolumn{1}{c}{(1)}   &\multicolumn{1}{c}{(2)}   \\
\hline
Minority            &     -0.323***&               \\
                    &(-0.213,-0.412)   &               \\
African American    &               &     -0.432***\\
                    &               &(-0.312,-0.532)   \\
Hispanic/LatinX     &               &     0.7027\\
                    &               &(-0.670,0.818)   \\
\hline
 Mean Response (White)&        0.4   &        0.4   \\
\hline Gender       &         Yes   &         Yes   \\
Education Level     &         Yes   &         Yes   \\
Inquiry Order       &         Yes   &         Yes   \\
\hline Observations &      45,202   &      45,202   \\
\\[-1.8ex]\hline                          
\hline \\[-1.8ex]           
\end{tabular}      
\begin{tablenotes}[scriptsize,flushleft] 
\scriptsize                         
\item Notes: Table reports coeficients from a within-property linear regression model including controls for gender, education and order the inquiry was sent. Standard errors clustered at the CBSA Downtown/Suburb level. 90\% confidence intervals reported in parentheses.                        
\end{tablenotes}                         
\end{threeparttable}                         
\end{table}        
%

\pagebreak

\begin{figure}[H]
\caption{US Map} \label{fig:robust}
    \includegraphics[scale=0.75]{../views/fig1a.pdf}   
 \flushleft
 Note: This figure shows ...
\end{figure}

%%%%%%%%%%%%%%%%%%%%%%%%%%%%%%%%
%   APPENDIX	 Tables	        %
%%%%%%%%%%%%%%%%%%%%%%%%%%%%%%%%
\pagebreak
\appendix
\renewcommand{\theequation}{\Alph{chapter}.\arabic{equation}}

\setcounter{figure}{0}
\setcounter{table}{0}
\makeatletter 
\renewcommand{\thefigure}{A.\@arabic\c@figure}
\renewcommand{\thetable}{A.\@arabic\c@table}

\section{Appendix: Tables and Figures}\label{sec:appendix_tables} 

\end{document}
